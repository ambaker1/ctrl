\documentclass{article}

% Input packages & formatting
% Packages

% Math packages
\usepackage{amsmath} % Extended math functions
\usepackage{amssymb} % Extended math symbols (loads in amsfonts)
\usepackage{bm} % Bold math symbols
\usepackage{mathtools}

% Figure packages
\usepackage{caption} % Caption formatting for university standard
\usepackage{graphicx} % includegraphics command
\usepackage{subcaption} % Subfigures
\usepackage[section]{placeins} % Place floats in section
\usepackage{wrapfig}

% Table packages
\usepackage{booktabs} % Better tables
\usepackage{bigstrut} % Merged table cells
\usepackage{longtable} % Tables which overflow into next page
\usepackage{array}
\usepackage{colortbl} % Color table cells
\usepackage{makecell}
\usepackage{multirow}

% Fonts
\usepackage{lmodern} % Use latin modern rather than computer modern. Better for font encoding.
\usepackage[T1]{fontenc} % Allow text to be searchable in output

% Other packages
\usepackage{appendix} % Appendix environment
\usepackage{nextpage} % Cleartooddpage command
%\usepackage[square,comma,sort,numbers]{natbib} % Reference formatting
\usepackage{setspace} % Line spacing
\usepackage{listings} % Display code with syntax highlighting
\usepackage{upquote} % Vertical quotes in verbatim
\usepackage{xcolor} % Colors
\usepackage{titlesec} % Header spacing
\usepackage{xparse} % for tcolorbox
\usepackage[listings]{tcolorbox} % Colored boxes for highlighting syntax
\tcbuselibrary{breakable}
\tcbuselibrary{skins}
\usepackage{enumitem} % better enumerate/itemize options
\usepackage{fancyhdr}
\usepackage{multicol}
\usepackage{ifthen}
\usepackage{xstring}

% Table of contents
\usepackage{imakeidx} % Index page
\usepackage{tocloft} % Control of table of contents
\usepackage[nottoc]{tocbibind} % Adds bibliography, table of tables, table of figures, to table of contents
\usepackage[bookmarks,linktocpage,hidelinks]{hyperref} % Hyperlinks for sections, figures, etc.

% Formatting
% Page format
\setlength{\oddsidemargin}{0.00in}  % Left side margin for odd numbered pages
\setlength{\evensidemargin}{0.00in} % Right side margin for even numbered pages
\setlength{\topmargin}{0.00in}      % Top margin
\setlength{\headheight}{0.20in}     % Header height
\setlength{\headsep}{0.20in}        % Separation between header and main text
\setlength{\topskip}{0.00in}        % Top skip
\setlength{\textwidth}{6.50in}      % Width of the text
\setlength{\textheight}{8.50in}     % Height of the text
\setlength{\footskip}{0.50in}       % Foot skip
\setlength{\parindent}{0.00in}      % First line indentation
\setlength{\parskip}{6pt}        % Space between two paragraphs

% Captions (figures, tables, etc.)
\setlength{\floatsep}{\parskip}          % Space left between floats.
\setlength{\textfloatsep}{\floatsep}   % Space between last top float
% or first bottom float and the text
\setlength{\intextsep}{\floatsep}      % Space left on top and bottom
% of an in-text float
\setlength{\abovecaptionskip}{0.1in plus 0.25in}  % Space above caption
\setlength{\belowcaptionskip}{0.1in plus 0.25in}  % Space below caption
\setlength{\captionmargin}{0.50in}     % Left/Right margin for caption
\setlength{\abovedisplayskip}{0.00in plus 0.25in} % Space before Math stuff
\setlength{\belowdisplayskip}{0.00in plus 0.25in} % Space after Math stuff
\setlength{\arraycolsep}{0.10in}       % Gap between columns of an array
\setlength{\jot}{0.10in}                % Gap between multiline equations
\setlength{\itemsep}{0.10in}           % Space between successive items

% Counters (no section numbering)
\setcounter{tocdepth}{3}
\setcounter{secnumdepth}{0}

% Spacing
\setstretch{1.5}

\titlespacing*{\section}{0cm}{6pt}{6pt}[0cm]
\titlespacing*{\subsection}{0cm}{6pt}{6pt}[0cm]
\titlespacing*{\subsubsection}{0cm}{6pt}{6pt}[0cm]

\titleformat{\section}
{\sffamily\huge}{}{0pt}{\titlerule\vspace{-0.2cm}}
\titleformat{\subsection}
{\sffamily\itshape\Large}{}{0pt}{}

% Macro for syntax
\newtcolorbox{syntax}{
    size=small,
    sharp corners,
    colframe=black,
    colback=yellow,
    fontupper=\bfseries\ttfamily
}

% Macro for argument table
\newenvironment{args}{
    \begin{tabular}{>{\bfseries\ttfamily}p{0.25\linewidth} p{0.69\linewidth}}
    }{
    \end{tabular}\par
    \vspace{0.5\baselineskip}
}

% Note: Requires packages "listing", "xcolor", and "textcomp"
\lstdefinelanguage{verbatim}{
    basicstyle=\ttfamily\small,
    xleftmargin=9pt,
    xrightmargin=9pt,
    columns=fullflexible,
    keepspaces=true,
    comment=[l]{\#},
    breaklines=true
}

\lstdefinestyle{verbatim}{
    commentstyle=\color{gray},
}


% Example code
\AtBeginDocument{
\newtcolorbox[blend into=listings]{example}[2][]{
    colback=blue!3!white,
    colframe=black,
    colbacktitle=blue!15!white,
    coltitle=black,
    sharp corners,
    enhanced,
    breakable,
    size=small,
    before upper={
        \setstretch{1.0}\lstset{language=verbatim,style=verbatim}\vspace{3pt}\textsf{\textit{Code:}}
    },
    subtitle style={
        colback=blue!20!white,
        fonttitle=\sffamily
    },
    before lower={
        \setstretch{1.0}\lstset{language=verbatim,style=verbatim}\vspace{3pt}\textsf{\textit{Output:}}
    },
    fonttitle=\sffamily,
    title={#2},
    #1
}
}

% Links to sub and subsub commands - optional boolean argument, default true. if false, only displays subcmd.

% Commands (and command ensembles)
\newcommand{\command}[1]{\protect\hypertarget{#1}{#1}\index{#1}}
\newcommand{\subcommand}[2]{\protect\hypertarget{#1 #2}{#1 #2}\index{#1!#2}}
\newcommand{\cmdlink}[1]{\protect\hyperlink{#1}{\textit{#1}}}
\newcommand{\subcmdlink}[3][1]{\protect\hyperlink{#2 #3}{\ifnum#1=1\relax\textit{#2 #3}\else\textit{#3}\fi}}

% Methods (first arg is class)
\newcommand{\method}[2]{\protect\hypertarget{$#1Obj #2}{\$#1Obj #2}\index{#1 methods!#2}}
\newcommand{\methodlink}[3][1]{\protect\hyperlink{$#2Obj #3}{\ifnum#1=1\relax\textit{\$#2Obj #3}\else\textit{#3}\fi}}

% Macros for figure/table names
\newcommand{\fig}{\figurename\ }
\newcommand{\figs}{\figurename s }
\newcommand{\tbl}{\tablename\ }
\newcommand{\tbls}{\tablename s }
\newcommand{\eq}{Eq. }
\newcommand{\eqs}{Eqs. }
\renewcommand{\lstlistingname}{Example}% Listing -> Example
\renewcommand{\lstlistlistingname}{List of \lstlistingname s}% List of Listings -> List of Examples
\newcommand{\ex}{Example }
\newcommand{\exs}{Examples }
\newcommand{\var}[1]{\texttt{\textbf{\$#1}}}

% Header/footer
\renewcommand{\headrulewidth}{0pt}

% Changes to hyperlinks (URLs)
\renewcommand\UrlFont{\color{blue}\rmfamily}

% New column type 
% https://tex.stackexchange.com/questions/75717/how-can-i-mix-itemize-and-tabular-environments
\newcolumntype{L}{>{\labelitemi~~}l<{}}
\newcommand{\version}{0.9}

\renewcommand{\cleartooddpage}[1][]{\ignorespaces} % single side
\newcommand{\caret}{$^\wedge$}

% Other macros
\renewcommand{\^}[1]{\textsuperscript{#1}}
\renewcommand{\_}[1]{\textsubscript{#1}}

\title{\Huge Tcl Variable Utilities\\\small Version \version}
\author{Alex Baker\\\small\url{https://github.com/ambaker1/vutil}}
\date{\small\today}
\begin{document}
\maketitle
\begin{abstract}
\begin{center}
This package provides various utilities for working with variables in Tcl.
\end{center}
\end{abstract}
\clearpage
\section{Printing Variables to Screen} 
The \cmdlink{pvar} command is a short-hand function for printing the name and values of Tcl variables.
\begin{syntax}
\command{pvar} \$name1 \$name2 …
\end{syntax}
\begin{args}
\$name1 \$name2 … & Name(s) of variables to print
\end{args}

\begin{example}{Printing variables to screen}
\begin{lstlisting}
set a 5
set b 7
set c(1) 5
set c(2) 6
pvar a b c
\end{lstlisting}
\tcblower
\begin{lstlisting}
a = 5
b = 7
c(1) = 5
c(2) = 6
\end{lstlisting}
\end{example}
\clearpage
\section{Initializing Local Namespace Variables}
The command \cmdlink{local} is the counterpart to the Tcl \textit{global} command, and creates local variables linked to variables in the current namespace, by simply calling the Tcl \textit{variable} command multiple times.
\begin{syntax}
\command{local} \$name1 \$name2 …
\end{syntax}
\begin{args}
\$name1 \$name2 … & Name(s) of variables to initialize
\end{args}
\begin{example}{Access namespace variables in a procedure}
\begin{lstlisting}
# Define global variables
global a b c
set a 1
set b 2
set c 3
namespace eval ::foo {
    # Define local variables
    local a b c
    set a 4
    set b 5
    set c 6
}
proc ::foo::bar1 {} {
    # Access global variables
    global a b c
    list $a $b $c
}
proc ::foo::bar2 {} {
    # Access local variables
    local a b c
    list $a $b $c
}
puts [::foo::bar1]; # global a b c
puts [::foo::bar2]; # local a b c
\end{lstlisting}
\tcblower
\begin{lstlisting}
1 2 3
4 5 6
\end{lstlisting}
\end{example}

\clearpage

\section{Default Values}
The command \cmdlink{default} assigns values to variables if they do not exist.

\begin{syntax}
\command{default} \$varName \$value
\end{syntax}
\begin{args}
\$varName & Name of variable to set \\
\$value & Default value for variable
\end{args}

The example below shows how default values are only applied if the variable does not exist.

\begin{example}{Variable defaults}
\begin{lstlisting}
set a 5
default a 7
puts $a
unset a
default a 7
puts $a
\end{lstlisting}
\tcblower
\begin{lstlisting}
5
7
\end{lstlisting}
\end{example}
\clearpage
\section{Variable Locks}
The command \cmdlink{lock} uses Tcl variable traces to make a read-only variable.

\begin{syntax}
\command{lock} \$varName <\$value>
\end{syntax}
\begin{args}
\$varName & Variable name to lock. \\
\$value & Value to lock variable at. Default self-locks (uses current value).
\end{args}

The command \cmdlink{unlock} unlocks previously locked variables so that they can be modified again.

\begin{syntax}
\command{unlock} \$name1 \$name2 …
\end{syntax}
\begin{args}
\$name1 \$name2 … & Variables to unlock.
\end{args}

\begin{example}{Variable locks}
\begin{lstlisting}
lock a 5
set a 7
puts $a
unlock a
set a 7
puts $a
\end{lstlisting}
\tcblower
\begin{lstlisting}
5
7
\end{lstlisting}
\end{example}

\clearpage

\section{Variable-Object Ties}
As of Tcl version 8.6, there is no garbage collection for Tcl objects, they have to be removed manually with the ``destroy'' method. 
The command \cmdlink{tie} is a solution for this problem, using variable traces to destroy the corresponding object when the variable is unset or modified. 
Tie is separate from lock; a tie will override a lock, and a lock will override a tie.
\begin{syntax}
\command{tie} \$varName <\$object>
\end{syntax}
\begin{args}
\$varName & Variable name to tie to object. \\
\$object & Object to tie variable to. Default self-ties (uses current value).
\end{args}

In similar fashion to \cmdlink{unlock}, tied variables can be untied with the command \cmdlink{untie}.

\begin{syntax}
\command{untie} \$name1 \$name2 …
\end{syntax}
\begin{args}
\$name1 \$name2 … & Variables to untie.
\end{args}

\begin{example}{Variable-object ties}
\begin{lstlisting}
oo::class create foo {
    method hi {} {
        puts hi
    }
}
tie a [foo create bar]
set b $a; # alias variable
unset a; # triggers ``destroy''
$b hi; # throws error
\end{lstlisting}
\tcblower
\begin{lstlisting}
invalid command name "::bar"
\end{lstlisting}
\end{example}
\end{document}
